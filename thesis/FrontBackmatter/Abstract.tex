%*******************************************************
% Abstract
%*******************************************************
%\renewcommand{\abstractname}{Abstract}
\pdfbookmark[1]{Abstract}{Abstract}
% \addcontentsline{toc}{chapter}{\tocEntry{Abstract}}
\begingroup
\let\clearpage\relax
\let\cleardoublepage\relax
\let\cleardoublepage\relax

\chapter*{Abstract}
%Short summary of the contents in English\dots a great guide by
%Kent Beck how to write good abstracts can be found here:
%\begin{center}
%\url{https://plg.uwaterloo.ca/~migod/research/beckOOPSLA.html}
%\end{center}

%In sum, the work we present here has shown that convolutional neural networks provide state-of-the-art performance in denoising applications when given appropriate data; the network architecture being secondary as most networks we experimented with learned to denoise effectively (albeit some more slowly). We also developed a working framework to train conditional generative adversarial networks for image denoising; these networks can potentially obtain greater performance with more difficult subjects. We believe this information, along with the \acl{NIND} we present here, will substantially improve the field of image denoising and lead to the development of more efficient techniques that can be used in digital image processing applications.

Convolutional neural networks have been the focus of research aiming to solve image denoising problems but their performance remains unsatisfactory for most applications. These networks are trained with synthetic noise distributions that do not accurately reflect the noise captured by image sensors. Some datasets of clean-noisy image pairs have been introduced but they are usually meant for benchmarking or specific applications. We introduce the Natural Image Noise Dataset (NIND), a dataset of DSLR-like images with varying levels of ISO noise that is large enough to train models for blind denoising over a wide range of noise. We demonstrate the use of denoising models trained with the NIND and show that they significantly outperform BM3D on ISO noise from unseen images, even when generalizing to images from a different type of camera. We expect that this dataset will prove useful for future image denoising applications. In support of this, we investigate the use of conditional generative adversarial networks (cGANs), which provide better denoising performance in some edge cases that a paired dataset may not cover. This has enabled us to develop a framework that lays the groundwork for training (c)GANs for image denoising purposes, as well as introduce a pair of discriminator-generator architectures that perform well in such systems. This work provides proof of concept support for the use of deep learning for photographic image denoising, as facilitated by the NIND.

\vfill

%\begin{otherlanguage}{ngerman}
%\pdfbookmark[1]{Zusammenfassung}{Zusammenfassung}
%\chapter*{Zusammenfassung}
%Kurze Zusammenfassung des Inhaltes in deutscher Sprache\dots
%\end{otherlanguage}

\endgroup

\vfill
