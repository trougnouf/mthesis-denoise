\chapter{Generative Adversarial Networks}

Generative Adversarial Networks are pairs of competing networks; a generator and a discriminator contest each other in a zero-sum game. The goal of the generator is to fool the discriminator by generating content which cannot reliably be differentiated from real ground-truth content. The discriminator attempts to tell generated content apart from real data. Each network is trained as the other's loss function (or one thereof).

The main advantage of a \acl{GAN} is that the loss function is learned; it does not need to be defined and it can exceed the performance obtained with a conventional loss that is a limiting factor. The generator can learn to restore high-frequency details which cannot be determined from the noisy observation alone (and that are not necessarily visible in the ground-truth) instead of averaging the many possible solutions.

Moreover, data may be added without a matching pair in some scenarios. This does not work with \acl{cGAN} nor can it be combined with other loss functions (combining losses is often necessary to ensure that the result matches and that the details are not overdone), but it can be of great usefulness to further train an existing network to perform better on specialized tasks.

\section{Types of GANs}

\begin{itemize}
  \item \ac{GAN}
  A \ac{GAN} is the simplest approach introduced in \cite{gan}. A generator is called with a given input, then the discriminator trains on the data generated by the generator as well as another mini-batch of clean data. 
  
  
  While it may seem counter-intuitive to train the discriminator with mini-batches of only one label, this phenomenon may be explained with the use of Batch Normalization which performs better with a \cite{gantechniques}
  and its running statistics on the given batch which perform better
  \item \ac{cGAN}
  
\end{itemize}
CycleGAN, does not apply here because

\section{Chosen GAN}
cGAN and GAN
PatchGAN
The generator is the same UNet architecture used in previous

TODO Move this to later discussion


% should investigate: https://arxiv.org/abs/1803.07422 (Patch-Based Image Inpainting with Generative Adversarial Networks)
